\documentclass{res}
\addtolength{\oddsidemargin}{-.2in}
\addtolength{\topmargin}{-.5in}
\setlength{\textheight}{11in}

\begin{document} 
\small
\name{FINLAY MAGUIRE\\[12pt]}    

\address{\bf ACADEMIC ADDRESS\\Natural History Museum\\Cromwell Road\\London\\SW7 5BD\\+44 (0)20 7942 6322\\richardslab.exeter.ac.uk}
\address{\bf HOME ADDRESS\\7 Albion Place\\Exeter\\EX4 6LH\\+44 (0)77 6989 5717\\root@finlaymagui.re\\finlaymagui.re}

\begin{resume}
        
\section{EDUCATION}          

\vspace{-0.05in}
\begin{tabbing}
   \hspace{2in}\= \hspace{2.6in}\= \kill 
    {\bf Doctor of Philosophy (To be completed)}\\ 
    {\bf Natural History Museum}\\
    {\bf University College London
    } \>      \>October 2011 - Present\\

   \end{tabbing}\vspace{-20pt}  
   
   Joint institution PhD in Molecular Evolution/Bioinformatics on the reconstruction of cellular interactions in a nascent endosymbiotic system.  This resarch is co-supervised by Professor Max Telford (EMBO YIP 2000) at University College London and Dr Thomas Richards (EMBO YIP 2012) at the Natural History Museum.  Project is currently being completed as a sitting scientist at the University of Exeter where Dr Richard's lab is now located.

\vspace{-0.1in}
 \begin{tabbing}
   \hspace{2in}\= \hspace{2.6in}\= \kill 
     {\bf Bachelor of Arts (Honours) Natural Science - Biological Sciences}\\
    {\bf Oxford University
    } \>      \>October 2008 - June 2011\\
                          

   \end{tabbing}\vspace{-20pt}  
Honours schools: Cellular \& Developmental Biology and the Biology of Plant \& Animal Disease\\
During this degree I conducted the following independent research:
\begin{itemize}
        \vspace{-10pt}
    \item The evolution of the folate biosynthesis gene fusions in the eukaryotes (supervised by Dr Thomas Richards) (now published - see publications section below for full citation).
    \item Explaining the functional diversity of leucine rich repeats (supervised by Dr Adrian Smith).
    \item A proposal to investigate the exchange of proteins between  \emph{Igniococcus hospitalis} and  \emph{Nanoarchaeum equitans} (supervised by Professor Ian Moore).
\end{itemize}


\section{SHORT TERM PROJECTS}

   \vspace{-0.05in} 

   \begin{tabbing}
   \hspace{2in}\= \hspace{2.6in}\= \kill 
   {\bf National Data Science Bowl} \> \>        December 2014 - March 2015\\
     \end{tabbing}\vspace{-20pt}      
     Participating in a Kaggle machine learning competition focussed on image classification of 121 classes of plankton. Formed part of 3-person team with PhD students from University of Edinburgh. Our strategy is to use a CUDA-accelerated convnet implemented in theano.

   \vspace{-0.1in} 
   \begin{tabbing}
   \hspace{2in}\= \hspace{2.6in}\= \kill 
   {\bf American Epilepsy Society Seizure Prediction Challenge} \> \>        August - October 2014\\
     \end{tabbing}\vspace{-20pt}      
     Participated in a Kaggle machine learning competition trying to differentiate pre-seizure iEEGs from normal brain activity as part of 3-person team with PhD students from University of Edinburgh. 
     Placed in the top 5\% of teams (16/527) using an ensemble of random forest models trained on several features such as multivariate autoregression models and frequency power bands.

   \vspace{-0.1in} 
   \begin{tabbing}
   \hspace{2in}\= \hspace{2.6in}\= \kill 
    {\bf NASA Planetary Biology Program} \> \>        July - October 2013\\
     \end{tabbing}\vspace{-20pt}      
     Funded short term research fellowship (10 weeks) with Dr David J. Smith at the Space Life Science Laboratories of NASA's John F. Kennedy Space Centre working on the MIST project. This project involved investigating transcriptomic adaptations of \textit{Bacillus} species to stratospheric conditions as well as sampling microbes in the stratosphere.

     \vspace{-0.1in}
   \begin{tabbing}
   \hspace{2in}\= \hspace{2.6in}\= \kill 
    {\bf BBC Cloud Lab Documentary} \> \>        October 2013\\
     \end{tabbing}\vspace{-20pt}      
     Acted as NASA representative and scientific adviser for the British Broadcasting Company (BBC) `Cloud Lab' documentary on atmospheric sciences. In this role I created microbial sampling and exposure protocols as well assembling the necessary materials and airship fixtures to conduct these experiments. I additionally briefed Dr Jim McQuaid (University of Leeds) and producers on the relevant microbiological background. 

   \vspace{-0.1in}
      \begin{tabbing}
   \hspace{2in}\= \hspace{2.6in}\= \kill 
    {\bf Society for General Microbiology Harry Smith Vacation Studentship} \> \>        June - August 2010\\
     \end{tabbing}\vspace{-20pt}      
     Funded summer research project (9 weeks) with Dr Thomas Richards at the University of Exeter on the evolution of folate biosynthesis gene fusions in the eukaryotes.

\vspace{-0.1in}
   \begin{tabbing}
   \hspace{2in}\= \hspace{2.6in}\= \kill 
    {\bf BioMAP Egypt} \> \>        July - August 2008\\
     \end{tabbing}\vspace{-20pt}      
     Research assistant (5 weeks) on the BioMAP GIS project monitoring biodiversity in the Sinai desert in association with the University of Nottingham and the Suez Canal University.

\section{AWARDS AND GRANTS}

   \vspace{-0.05in}  

  \begin{tabbing}
   \hspace{2in}\= \hspace{2.6in}\= \kill 
    {\bf NASA Planetary Biology Program} \> \>July - October 2013\\
                            \> 
                            \> John F. Kennedy Space Centre\\

   \end{tabbing}\vspace{-30pt}     
   Competitive award which funded a short term (10 week) research fellowship \\ with Dr David J. Smith at NASA Kennedy Space Centre on bacterial adaptations \\ to extreme conditions (\textsterling3500).

   \vspace{-0.1in}  
 
   
   \begin{tabbing}
   \hspace{2in}\= \hspace{2.6in}\= \kill 
    {\bf Earth and Space Foundation Exploration Award} \> \> 2013\\
                            \> 
                            \> John F. Kennedy Space Centre\\


   \end{tabbing}\vspace{-30pt}     
   Awarded for work conducted at NASA Kennedy Space Centre with Dr David J. \\ Smith in the use of terrestrial environments to advance the exploration of space (\textsterling300).

   \vspace{-0.1in}  
   
   \begin{tabbing}
   \hspace{2in}\= \hspace{2.6in}\= \kill 
    {\bf FEMS Young Scientist Meeting Grant} \> \> 19th-24th October 2013\\
                            \>  
                            \> Sant Feliu de Guixols, Spain \\
                       
   \end{tabbing}\vspace{-30pt}      

   Federation of European Microbiological Societies (FEMS) grant to attend\\ and present at the European Molecular Biology Organisation (EMBO) conference:\\ Comparative Genomics of Eukaryotic Microorganisms 2013 (\textsterling350).

 
 \vspace{-0.1in}  
   \begin{tabbing}
   \hspace{2in}\= \hspace{2.6in}\= \kill 
    {\bf FEMS Young Scientist Meeting Grant} \> \> 15th-20th October 2011\\
                            \>  
                            \> Sant Feliu de Guixols, Spain \\
                       
   \end{tabbing}\vspace{-30pt}      

   Federation of European Microbiological Societies (FEMS) grant to attend\\ and present at the European Molecular Biology Organisation (EMBO) conference: \\ Comparative Genomics of Eukaryotic Microorganisms 2011 (\textsterling350).

 
\vspace{-0.1in}  
   \begin{tabbing}
   \hspace{2in}\= \hspace{2.6in}\= \kill
    {\bf Society for General Microbiology Undergraduate Student Grant} \> \> 5th-7th September 2011\\
                            \> 
                            \> University of York \\

   \end{tabbing}\vspace{-30pt}      
   Grant for travel and accommodation to attend and present at the Society for \\ General Microbiology (SGM) Summer 2011 conference at the University of York (\textsterling170). 

 
   \begin{tabbing}
   \hspace{2in}\= \hspace{2.6in}\= \kill 
    {\bf Dukinfield Exhibition in Biological Sciences} \> \>        October 2009-July 2011\\
                                               
                                               \> \> Somerville College, Oxford \\
   \end{tabbing}\vspace{-30pt}     
   Awarded and maintained throughout my undergraduate degree for \\academic excellence (\textsterling300).

\vspace{-0.1in}  
   \begin{tabbing}
   \hspace{2in}\= \hspace{2.6in}\= \kill 
    {\bf Society for General Microbiology Harry Smith Vacation Studentship} \> \>June 2010-August 2010\\
                            \> 
                            \> University of Exeter\\
   \end{tabbing}\vspace{-30pt}     
      Award funding a summer research project with Dr Thomas Richards \\
       at the University of Exeter on the evolution of folate biosynthesis gene \\
       fusions in the eukaryotes (\textsterling1890).

\section{PUBLICATIONS} 
\vspace{-0.05in}
 \begin{tabbing}
   \hspace{2.3in}\= \hspace{2.6in}\= \kill
   {\bf ``Complex patterns of gene fission in the eukaryotic folate biosynthesis pathway.''\\
  }
   \end{tabbing}\vspace{-20pt}
   Maguire, F., Henriquez, F.L., Leonard, G., Dacks, J.B., Brown, M.W.,  Richards, T.A.\\
   Genome Biology and Evolution 6, 2709-2720, 2014\\

\vspace{-0.2in}
\begin{tabbing}
   \hspace{2.3in}\= \hspace{2.6in}\= \kill
   {\bf ``Organelle Evolution: A Mosaic of `Mitochondrial' Functions''} 
   \end{tabbing}\vspace{-20pt}
    Maguire, F., Richards, T.A.\\ 
    Current Biology 24(11), R518-R520, 2014\\

\vspace{-0.2in}
\begin{tabbing}
   \hspace{2.3in}\= \hspace{2.6in}\= \kill
   {\bf ``Diverse molecular signatures for `active' Perkinsea in marine sediments''} 
   \end{tabbing}\vspace{-20pt}
     Chambouvet, A., Berney, C., Romac, S., Audic, S., Maguire, F., de Vargas, C., Richards, T.A.\\
     BMC microbiology 14(1), 110, 2014\\

\vspace{-0.2in}
\begin{tabbing}
   \hspace{2.3in}\= \hspace{2.6in}\= \kill
   {\bf ``A Balloon-Based Payload for Exposing Microorganisms in the Stratosphere (E-MIST)''}
   \end{tabbing}\vspace{-20pt}
Smith, D.J., Thakrar, P.J., Bharrat, A.E., Dokos, A.G., Kinney, T.L., James, L.M., Khodadad, C.L., Maguire, F., Maloney, P.R., Dawkins, N.L.\\
Gravitational and Space Research 2 (2), 2014\\

\vspace{-0.1in}
\begin{tabbing}
   \hspace{2.3in}\= \hspace{2.6in}\= \kill
   {\bf ``Cryptic infection of Perkinsea protists of diverse and globally distributed tadpoles''}
   \end{tabbing}\vspace{-20pt}
   Chambouvet, A., Gower, D.J., Jirku, M., Yabsley, M.J., Leonard, G., Maguire, F., Bittencourt, G., Wilkinson, M., Richards, T.A.\\
\emph{Manuscript in review.}



\section{CONFERENCES AND WORKSHOPS}
  \vspace{-0.05in}
    
\begin{tabbing}
\hspace{2in}\= \hspace{2.6in}\= \kill
{\bf EMBO YIP PhD Course} \> \> 2nd-7th December 2013\\
\> \> EMBL, \\
\> \> Heidelberg, Germany\\
\end{tabbing}\vspace{-40pt}
European Molecular Biology Organisation Young Investigator \\
PhD course participant.
\newpage
\vspace{-0.1in}
     \begin{tabbing} 
   \hspace{2in}\= \hspace{2.6in}\= \kill 
    {\bf EMBO: Comparative Genomics of Eukaryotic Microorganisms}
    \>  \> 19th-24th October 2013\\
                         \>    \> Sant Feliu de Guixols, Spain
   \end{tabbing}\vspace{-20pt}      
      Poster presentation: ``Key endosymbiont proteins implicated in the \\ maintenance of the photosynthetic endosymbiosis between \textit{Paramecium bursaria} \\ and \textit{Chlorella}''.
      \vspace{-0.1in}

\begin{tabbing}
\hspace{2in}\= \hspace{2.6in}\= \kill
{\bf Molecular Evolution Workshop} \> \> 22nd July - 1st August 2012\\
\> \> MBL, Woods Hole \\
\> \> MA, USA \\
\end{tabbing}\vspace{-40pt}
Workshop participant on this competitive and internationally renowned course.
  
   \vspace{-0.1in}
   \begin{tabbing}
   \hspace{2in}\= \hspace{2.6in}\= \kill 
   {\bf National Association for Research in Science Teaching } \>  \> 25th-28th March 2012\\
                        \>     \> Indianapolis, IN, USA
   \end{tabbing}\vspace{-20pt}      
Oral presentation: ``Working on the Public's Perception and Understanding \\of Science and Scientists
through a Popular, Open-access `AskScience' Website'' \\
de la Rubia, L.A., Marus, S., Maguire, F. \\
(\emph{Presented in absentia}).
   
   
   \vspace{-0.1in}
    \begin{tabbing}
   \hspace{2in}\= \hspace{2.6in}\= \kill 
    {\bf Tennessee Maths and Science Education Research Conference} \>  \> 2nd-3rd February 2012\\
                        \>     \> Murfreesboro, TN, USA
   \end{tabbing}
   
  \vspace{-30pt}     
Oral presentation: ``Online Conversations as a Way of Understanding the \\Public's Views of the Nature of Science: Research on Reddit's `AskScience' '' 
 \\ de la Rubia, L.A., Marus, S., Maguire, F. \\
 (\emph{Presented in absentia})
  \vspace{-0.1in}
  
  \begin{tabbing}
   \hspace{2in}\= \hspace{2.6in}\= \kill 
    {\bf Systematics Association: Young Systematists Forum} \>  \> 1st December 2011\\
                        \>     \> Natural History Museum
   \end{tabbing}\vspace{-20pt}      
   
   Poster presentation: ``Folate Biosynthesis Gene Fusions Evolution \\in the Eukaryotes''.
    \vspace{-0.1in}  
    \begin{tabbing}
   \hspace{2in}\= \hspace{2.6in}\= \kill 
    {\bf Society for General Microbiology Summer Conference} \> \>5th-7th September 2011\\
                          \>   \> University of York, UK

   \end{tabbing}\vspace{-20pt}      
      Poster presentation: ``Evolution of Folate Biosynthesis Gene Fusions \\in the Eukaryotes''.
      
  
  \vspace{-0.1in}
     \begin{tabbing} 
   \hspace{2in}\= \hspace{2.6in}\= \kill 
    {\bf EMBO: Comparative Genomics of Eukaryotic Microorganisms}
    \>  \> 15th-20th October 2011\\
                         \>    \> Sant Feliu de Guixols, Spain
   \end{tabbing}\vspace{-20pt}      
      Poster presentation: ``Evolution of Folate Biosynthesis Gene Fusions \\in the Eukaryotes''.

  \section{TEACHING AND SUPERVISION}
  \begin{itemize}
          \vspace{-1pt}
      \item Wellcome Trust ISSF Genomics Course (February 4th-25th 2015)
      \item Wellcome Trust Biomedical Informatics Hub Unix and Perl Course (October 2014-July 2015)
      \item Co-supervision of final year undergraduate project at University of Exeter identifying components of the RNAi system in \textit{Paramecium bursaria} (November-December 2014)
      \item Wellcome Trust Biomedical Informatics Hub Image Processing with Python Workshop (18th-19th September 2014)
      \item Software Carpentry Exeter Bootcamp (14th-15th November 2013)
      \item Supervision of a visiting summer research student registered at the University of Oxford. The project involved bioinformatic analysis of arsenate resistance HGTs. (June-October 2013)
  \end{itemize}

 \section{OUTREACH}          
 \vspace{5pt}
Moderator of the ``AskScience'' science education forum with over 15 million monthly views and 25,000 unique visitors per hour.
\vspace{-5pt}
 \section{TECHNICAL SKILLS}
          \vspace{-1pt}
 \begin{itemize}
\item Experienced with UNIX/Linux, Zsh/Bash, Python, Awk, Git, \LaTeX
\item Familiarity with R, C, C++, perl, matlab, high-performance linux system administration, statistics and machine learning
\item Rudimentary knowledge of Javascript, Nimrod, CSS, Verilog, Prolog, Go
\item Expert knowledge in bioinformatics: transcriptomics, molecular evolution/phylogenetics, genomics, metabolomics, sequence analysis, annotation
\item Experience with an array of basic molecular and microbiological techniques and methodologies: PCR, culturing, cloning, sequencing technologies, fluorescence microscopy.
    \end{itemize}
	
 \section{OTHER INTERESTS}
          \vspace{-1pt}
 \begin{itemize}
     \item Completed massively open online courses (MOOC) in: machine learning, mathematical logic, linear algebra, algorithms, databases, statistics, data analysis, and music theory.
     \item Technical co-founder of `Awedify' a start-up seeking revolutionise the spoken-word audio format.
     \item PADI Open Water Diver
     \item St. John's Ambulance First Aider
     \item RYA Level 3 Sailing
     \item Languages: German (intermediate), French (basic/intermediate)
     \item Musical instruments: French Horn and Bass Guitar
 \end{itemize}
\end{resume}
\end{document}
