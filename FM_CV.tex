\documentclass{res}
\addtolength{\oddsidemargin}{-.2in}
\addtolength{\topmargin}{-.5in}
\setlength{\textheight}{11in}

\begin{document} 
\small
\name{FINLAY MAGUIRE\\[12pt]}    

\address{\bf ACADEMIC ADDRESS\\Natural History Museum\\Cromwell Road\\London\\SW7 5BD\\+44 (0)20 7942 6322\\richardslab.exeter.ac.uk}
\address{\bf HOME ADDRESS\\7 Albion Place\\Exeter\\EX4 6LH\\+44 (0)77 6989 5717\\root@finlaymagui.re\\finlaymagui.re}

\begin{resume}
        
\section{EDUCATION}          

\vspace{-0.05in}
\begin{tabbing}
   \hspace{2in}\= \hspace{2.6in}\= \kill 
    {\bf Doctor of Philosophy (To be completed)}\\ 
    {\bf Natural History Museum}\\
    {\bf University College London
    } \>      \>October 2011 - Present\\

   \end{tabbing}\vspace{-20pt}  
   
Joint institution PhD in Molecular Evolution/Bioinformatics between Professor Max Telford at University College London and Dr. Thomas Richards at the Natural History Museum.  Project is investigating the deep branching relationships within the Eukaryotes and the metabolic basis of photosynthetic endosymbioses in cilliates via a combination of molecular and bioinformatic techniques. %consider breaking up into 2 smaller sentences
 
 
\vspace{-0.1in}
 \begin{tabbing}
   \hspace{2in}\= \hspace{2.6in}\= \kill 
    {\bf Bachelor of Arts (Honours) Biological Sciences}\\
    {\bf Oxford University
    } \>      \>October 2008 - June 2011\\
                          

   \end{tabbing}\vspace{-20pt}  
Honours schools: Cellular and Devopmental Biology and Biology of Plant and Animal Disease\\
Undergraduate research: \\
- The evolution of the folate biosynthesis gene fusions in the eukaryotes (supervised by Dr Thomas Richards).
\\-Explaining the functional diversity of leucine rich repeats (supervised by Dr Adrian Smith).
\\-A proposal to investigate the exchange of proteins between  \emph{Igniococcus hospitalis} and  \emph{Nanoarchaeum equitans} (supervised by: Professor Ian Moore).


\section{SHORT TERM PROJECTS}

   \vspace{-0.05in} 

   \begin{tabbing}
   \hspace{2in}\= \hspace{2.6in}\= \kill 
   {\bf American Epilepsy Society Seizure Prediction Challenge} \> \>        August - October 2014\\
     \end{tabbing}\vspace{-20pt}      
     Participated in a Kaggle Machine Learning competition as part of team composed of PhD students from University of Edinburgh, University College London and University of Oxford

   \vspace{-0.1in} 
   \begin{tabbing}
   \hspace{2in}\= \hspace{2.6in}\= \kill 
    {\bf NASA Planetary Biology Program} \> \>        July - October 2013\\
     \end{tabbing}\vspace{-20pt}      
     Short term research project with Dr. David J. Smith at the Space Life Science Laboratories of NASA's John F. Kennedy Space Center working on the MIST project. This project involved investigating transcriptomic adaptations of \textit{Bacillus} species to stratospheric conditions as well as sampling microbes in the stratosphere

     \vspace{-0.1in}
   \begin{tabbing}
   \hspace{2in}\= \hspace{2.6in}\= \kill 
    {\bf BBC Cloud Lab Documentary} \> \>        October 2013\\
     \end{tabbing}\vspace{-20pt}      
     Acted as NASA representative and scientific advisor for the British Broadcasting Company Cloud Lab documentary on atmospheric science. Prepared microbial sampling and exposure protocols and materials as well as advised on correct means to conduct these experiments.

   \vspace{-0.1in}
      \begin{tabbing}
   \hspace{2in}\= \hspace{2.6in}\= \kill 
    {\bf Society for General Microbiology Harry Smith Vacation Studentship} \> \>        June - August 2010\\
     \end{tabbing}\vspace{-20pt}      
Summer research project with Dr. Thomas Richards at Exeter University on the evolution of folate biosynthesis gene fusions in the eukaryotes.

\vspace{-0.1in}
   \begin{tabbing}
   \hspace{2in}\= \hspace{2.6in}\= \kill 
    {\bf BioMAP Egypt} \> \>        July - August 2008\\
     \end{tabbing}\vspace{-20pt}      
Research assistant on BioMAP GIS project monitoring biodiversity in the Sinai desert in association with the University of Nottingham and the Suez Canal University.

\section{AWARDS AND GRANTS}

   \vspace{-0.05in}  

   \begin{tabbing}
   \hspace{2in}\= \hspace{2.6in}\= \kill 
    {\bf NASA Planetary Biology Program} \> \>July - October 2013\\
                            \> 
                            \> John F. Kennedy Space Center\\

   \end{tabbing}\vspace{-30pt}     
   Funded a research project with Dr. David J. Smith at NASA Kennedy Space Center \\on bacterial adaptations to extreme conditions (\textsterling3500).

   \vspace{-0.1in}  
   \newpage
   \begin{tabbing}
   \hspace{2in}\= \hspace{2.6in}\= \kill 
    {\bf FEMS Young Scientist Meeting Grant} \> \> 19th-24th October 2013\\
                            \>  
                            \> Sant Feliu de Guixols, Spain \\
                       
   \end{tabbing}\vspace{-30pt}      

    Federation of European Microbiological Societies grant to attend\\ and present at European Molecular Biology Organisation: Comparative \\Genomics of Eukaryotic Microorganisms 2013 (\textsterling350).

 
 \vspace{-0.1in}  
   \begin{tabbing}
   \hspace{2in}\= \hspace{2.6in}\= \kill 
    {\bf FEMS Young Scientist Meeting Grant} \> \> 15th-20th October 2011\\
                            \>  
                            \> Sant Feliu de Guixols, Spain \\
                       
   \end{tabbing}\vspace{-30pt}      

    Federation of European Microbiological Societies grant to attend\\ and present at European Molecular Biology Organisation: Comparative \\Genomics of Eukaryotic Microorganisms 2011 (\textsterling350).

 
\vspace{-0.1in}  
   \begin{tabbing}
   \hspace{2in}\= \hspace{2.6in}\= \kill
    {\bf Society for General Microbiology Undergraduate Student Grant} \> \> 5th-7th September 2011\\
                            \> 
                            \> University of York \\

   \end{tabbing}\vspace{-30pt}      
 Grant for travel and accommodation to attend and present at the\\ Society for General Microbiology summer conference at the University\\ of York (\textsterling170). 

 
   \begin{tabbing}
   \hspace{2in}\= \hspace{2.6in}\= \kill 
    {\bf Dukinfield Exhibition in Biological Sciences} \> \>        October 2009-July 2011\\
                                               
                                               \> \> Somerville College, Oxford \\
   \end{tabbing}\vspace{-30pt}     
   Awarded for academic excellence (\textsterling300).

\vspace{-0.1in}  
   \begin{tabbing}
   \hspace{2in}\= \hspace{2.6in}\= \kill 
    {\bf Society for General Microbiology Harry Smith Vacation Studentship} \> \>June 2010-August 2010\\
                            \> 
                            \> University of Exeter\\
   \end{tabbing}\vspace{-30pt}     
       Funded a summer research project with Dr. Thomas Richards \\
       at Exeter University on the evolution of folate biosynthesis gene \\
       fusions in the eukaryotes (\textsterling1890).

\section{PUBLICATIONS} 
\vspace{-0.05in}
 \begin{tabbing}
   \hspace{2.3in}\= \hspace{2.6in}\= \kill
   {\bf ``Complex patterns of gene fission in the eukaryotic folate biosynthesis pathway.''\\
  }
   \end{tabbing}\vspace{-20pt}
   Maguire, F., Henriquez, F.L., Leonard, G., Dacks, J.B., Brown, M.W.,  Richards, T.A.\\
   Genome Biology and Evolution, 2014\\

\vspace{-0.2in}
\begin{tabbing}
   \hspace{2.3in}\= \hspace{2.6in}\= \kill
   {\bf ``Organelle Evolution: A Mosaic of `Mitochondrial' Functions''} 
   \end{tabbing}\vspace{-20pt}
    Maguire, F., Richards, T.A.\\ 
    Current Biology 24(11), R518-R520, 2014\\

\vspace{-0.2in}
\begin{tabbing}
   \hspace{2.3in}\= \hspace{2.6in}\= \kill
   {\bf ``Diverse molecular signatures for `active' Perkinsea in marine sediments''} 
   \end{tabbing}\vspace{-20pt}
     Chambouvet, A., Berney, C., Romac, S., Audic, S., Maguire, F., de Vargas, C., Richards, T.A.\\
     BMC microbiology 14(1), 110, 2014\\

\vspace{-0.2in}
\begin{tabbing}
   \hspace{2.3in}\= \hspace{2.6in}\= \kill
   {\bf ``Cryptic infection of Perkinsea protists of diverse and globally distributed tadpoles''}
   \end{tabbing}\vspace{-20pt}
   Chambouvet, A., Gower, D.J., Jirku, M., Yabsley, M.J., Leonard, G., Maguire, F., Bittencourt, G., Wilkinson, M., Richards, T.A.\\
\emph{Manuscript in preparation.}

\vspace{-0.1in}
\begin{tabbing}
   \hspace{2.3in}\= \hspace{2.6in}\= \kill
   {\bf ``A gene transfer event infers a recent microsporidian parasite radiation''}
   \end{tabbing}\vspace{-20pt}
Campbell, S.E., Hamilton, P.B., Maguire, F., Richards, T.A., Williams, B.A.P.\\
\emph{Manuscript in preparation.}

\vspace{-0.1in}
\begin{tabbing}
   \hspace{2.3in}\= \hspace{2.6in}\= \kill
   {\bf ``A passenger blimp experiment measuring the survival of bacterial endospores in the atmosphere''}
   \end{tabbing}\vspace{-20pt}
Smith, D.J.*, Maguire, F.*, Morford, M.A., Khodadad, C.L., Vasihampayan, P.A., Maloney, P.R., McQuaid, J.B., Venkateswaran, K.J..\\
\emph{Manuscript in preparation.}



\section{CONFERENCES AND WORKSHOPS}
  \vspace{-0.05in}
    
\begin{tabbing}
\hspace{2in}\= \hspace{2.6in}\= \kill
{\bf EMBO YIP PhD Course} \> \> 2nd-7th December 2013\\
\> \> EMBL, \\
\> \> Heidelberg, Germany\\
\end{tabbing}\vspace{-40pt}
European Molecular Biology Organisation Young Investigator \\
PhD course participant.
\newpage
\vspace{-0.1in}
     \begin{tabbing} 
   \hspace{2in}\= \hspace{2.6in}\= \kill 
    {\bf EMBO: Comparative Genomics of Eukaryotic Microorganisms}
    \>  \> 19th-24th October 2013\\
                         \>    \> Sant Feliu de Guixols, Spain
   \end{tabbing}\vspace{-20pt}      
      Poster presentation: ``Key endiosymbiont proteins implicated in the \\ maintenace of the photosynthetic endosymbiosis between \textit{Paramecium bursaria} \\ and \textit{Chlorella}''.
      \vspace{-0.1in}

\begin{tabbing}
\hspace{2in}\= \hspace{2.6in}\= \kill
{\bf Molecular Evolution Workshop} \> \> 22nd July - 1st August 2012\\
\> \> MBL, Woods Hole \\
\> \> MA, USA \\
\end{tabbing}\vspace{-40pt}
Workshop participant.
  
   \vspace{-0.1in}
   \begin{tabbing}
   \hspace{2in}\= \hspace{2.6in}\= \kill 
   {\bf National Association for Research in Science Teaching } \>  \> 25th-28th March 2012\\
                        \>     \> Indianapolis, IN, USA
   \end{tabbing}\vspace{-20pt}      
Oral presentation: ``Working on the Public's Perception and Understanding \\of Science and Scientists
through a Popular, Open-access `AskScience' Website'' \\
de la Rubia, L.A., Marus, S., Maguire, F. \\
(\emph{Presented in absentia}).
   
   
   \vspace{-0.1in}
    \begin{tabbing}
   \hspace{2in}\= \hspace{2.6in}\= \kill 
    {\bf Tennessee Maths and Science Education Research Conference} \>  \> 2nd-3rd February 2012\\
                        \>     \> Murfreesboro, TN, USA
   \end{tabbing}
   
  \vspace{-30pt}     
Oral presentation: ``Online Conversations as a Way of Understanding the \\Public's Views of the Nature of Science: Research on Reddit's `AskScience' '' 
 \\ de la Rubia, L.A., Marus, S., Maguire, F. \\
 (\emph{Presented in absentia})
  \vspace{-0.1in}
  
  \begin{tabbing}
   \hspace{2in}\= \hspace{2.6in}\= \kill 
    {\bf Systematics Association: Young Systematists Forum} \>  \> 1st December 2011\\
                        \>     \> Natural History Museum
   \end{tabbing}\vspace{-20pt}      
   
   Poster presentation: ``Folate Biosynthesis Gene Fusions Evolution \\in the Eukaryotes''.
    \vspace{-0.1in}  
    \begin{tabbing}
   \hspace{2in}\= \hspace{2.6in}\= \kill 
    {\bf Society for General Microbiology Summer Conference} \> \>5th-7th September 2011\\
                          \>   \> University of York, UK

   \end{tabbing}\vspace{-20pt}      
      Poster presentation: ``Evolution of Folate Biosynthesis Gene Fusions \\in the Eukaryotes''.
      
  
  \vspace{-0.1in}
     \begin{tabbing} 
   \hspace{2in}\= \hspace{2.6in}\= \kill 
    {\bf EMBO: Comparative Genomics of Eukaryotic Microorganisms}
    \>  \> 15th-20th October 2011\\
                         \>    \> Sant Feliu de Guixols, Spain
   \end{tabbing}\vspace{-20pt}      
      Poster presentation: ``Evolution of Folate Biosynthesis Gene Fusions \\in the Eukaryotes''.

  \section{TEACHING AND SUPERVISION}
  \begin{itemize}
      \item Unix and Perl Course (October 2014-July 2015)
      \item Image Processing with Python Workshop (18th-19th September 2014)
      \item Software Carpentry Exeter Bootcamp (14th-15th November 2013)
      \item Co-supervision of a visiting summer research student registered at the University of Oxford. The project involved bioinformatic analysis of aresenate resistance HGTs. (June-October 2013)
  \end{itemize}
 \section{OUTREACH}          
Moderator of the ``AskScience'' science education outreach forum with over 15 million monthly views.
 \section{TECHNICAL SKILLS}
 \begin{itemize}
\item Experienced with UNIX/Linux, Zsh/Bash, Python, Awk, Perl, Git, \LaTeX
\item Familiarity with R, C++, linux sysadmin, big data
\item Some rudimentary knowledge of CSS, Verilog, Prolog, Go
\item Expert on several bioinformatics areas: transcriptomics, molecular evolution/phylogenetics, genomics, metabolomics, sequence analysis, annotation
\item Experience with an array of basic molecular and microbiological techniques and methodologies: PCR, culturing, cloning, sequencing technologies, fluorescence microscopy.
    \end{itemize}
	
 \section{OTHER INTERESTS}
 \begin{itemize}
     \item PADI Open Water Diver
     \item St. John's Ambulance First Aider
     \item RYA Level 3 Sailing
     \item Languages: German (intermediate), French (basic/intermediate)
     \item Musical instruments: French Horn and Bass Guitar
 \end{itemize}
\end{resume}
\end{document}
